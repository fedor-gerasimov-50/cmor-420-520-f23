\documentclass{article}

\usepackage{amsmath}
\usepackage{graphicx}
\usepackage{lipsum} 

\title{An intro to \LaTeX}
\author{Jesse Chan and the CMOR 420/520 class}
\date{}

\begin{document}

\maketitle

\section{Introduction}
Content. Content. Content. Content. Content. Content. Content. Content. Content. Content. Content. Content. Content. Content. Content. Content. Content. Content. Content. Content. Content. Content. Content. Content. Content. Content. Content. Content. Content. Content. Content.

\subsection{Subsection 1}

\subsubsection{Even more subsection}

The math environment 
$$\log(x)e^{2y} \gamma \in A$$
is the same as the math environment
\[
\log(x)e^{2y} \gamma \in A
\]

\begin{align}
  f &= \log(x)e^{2y} \gamma \in A  \label{eq:log} \\
  g &= \sin(\pi x)\log(ab)e^{2e} \gamma \in B  \\
  h &= \log(c)e^{2y} (1 + 2 + x + d) \gamma \in C    
\end{align}

% Comments: other useful environments like "gather", "subequations", "bmatrix", etc.

I am referencing equation (\ref{eq:log}). \lipsum

\begin{figure}
  \centering
  \includegraphics[width=.4\textwidth]{image.png}
  \caption{This is Rice's logo.}
  \label{fig:rice}
\end{figure}

Figure \ref{fig:rice} is the Rice logo.

\begin{table}
  \centering
  \begin{tabular}{|c|ccc|}
    Column 1 & Col 2 & C3 & c4 \\
    Column 1 & Col 2 & C3 & c4
  \end{tabular}
\end{table}

\end{document}

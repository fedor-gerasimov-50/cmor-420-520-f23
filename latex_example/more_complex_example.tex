\documentclass{article}
\usepackage{graphicx}
\usepackage{amsmath}

\title{Simple example}
\author{Jesse Chan}
\date{}

\begin{document}

\maketitle

\section{Introduction}

This is a Section. I am going to reference the advection equation (\ref{eq:advection})

Here is an equation in the way
\begin{equation}
F = ma.
\end{equation}

Here are multiple equations:
\begin{align}
  F &= ma\\
  E &= mc^2\\
  y &= ax + b
\end{align}

Here are multiple unlabeled equations:
\begin{align*}
  F &= ma\\
  E &= mc^2\\
  y &= ax + b
\end{align*}

Figure \ref{fig:example_fig} has nothing to do with equation (\ref{eq:advection}).
\begin{figure}[!h]
\centering
\includegraphics[width=0.75\textwidth]{image.png}
\caption{This is an example image.}
\label{fig:example_fig}
\end{figure}

Table \ref{tab:example_table} is an example of a Table in \LaTeX.
\begin{table}[!h]
  \centering
  \begin{tabular}{|l|c|c|r|}
    \hline
    Row 1 & $E=mc^2$ & $\frac{1}{2}$ & $\nabla u$\\
    Row 2 & $ax = b$ & $\frac{2}{3}$ & $\nabla\cdot U$\\
    \hline
  \end{tabular}
  \caption{This is an example table.}
  \label{tab:example_table}
\end{table}

% This is a comment
% Other useful mathematical environments: gather, $, $$, \[\]

A single equation inline $ax+b$. A separate equation
$$
ax+b
$$
Another way of defining a separate equation
\[
F=ma
\]

\begin{equation}
\frac{\partial u}{\partial t} + \frac{\partial u}{\partial x} = 0.
\label{eq:advection}
\end{equation}

Here I am citing paper \cite{first_citation} and then \cite{second_citation}.

\begin{thebibliography}{9}
\bibitem{first_citation} First citation author. \textit{First citation title}
\bibitem{second_citation} Second citation author. ``First citation title''.
\end{thebibliography}

\end{document}
